\documentclass[10pt,a4paper,onecolumn]{article}
\usepackage{marginnote}
\usepackage{graphicx}
\usepackage{xcolor}
\usepackage{authblk,etoolbox}
\usepackage{titlesec}
\usepackage{calc}
\usepackage{tikz}
\usepackage{hyperref}
\hypersetup{colorlinks,breaklinks,
            urlcolor=[rgb]{0.0, 0.5, 1.0},
            linkcolor=[rgb]{0.0, 0.5, 1.0}}
\usepackage{caption}
\usepackage{tcolorbox}
\usepackage{amssymb,amsmath}
\usepackage{ifxetex,ifluatex}
\usepackage{seqsplit}
\usepackage{fixltx2e} % provides \textsubscript
\usepackage[
  backend=biber,
%  style=alphabetic,
%  citestyle=numeric
]{biblatex}
\bibliography{paper.bib}



% --- Page layout -------------------------------------------------------------
\usepackage[top=3.5cm, bottom=3cm, right=1.5cm, left=1.0cm,
            headheight=2.2cm, reversemp, includemp, marginparwidth=4.5cm]{geometry}

% --- Default font ------------------------------------------------------------
% \renewcommand\familydefault{\sfdefault}

% --- Style -------------------------------------------------------------------
\renewcommand{\bibfont}{\small \sffamily}
\renewcommand{\captionfont}{\small\sffamily}
\renewcommand{\captionlabelfont}{\bfseries}

% --- Section/SubSection/SubSubSection ----------------------------------------
\titleformat{\section}
  {\normalfont\sffamily\Large\bfseries}
  {}{0pt}{}
\titleformat{\subsection}
  {\normalfont\sffamily\large\bfseries}
  {}{0pt}{}
\titleformat{\subsubsection}
  {\normalfont\sffamily\bfseries}
  {}{0pt}{}
\titleformat*{\paragraph}
  {\sffamily\normalsize}


% --- Header / Footer ---------------------------------------------------------
\usepackage{fancyhdr}
\pagestyle{fancy}
\fancyhf{}
%\renewcommand{\headrulewidth}{0.50pt}
\renewcommand{\headrulewidth}{0pt}
\fancyhead[L]{\hspace{-0.75cm}\includegraphics[width=5.5cm]{/Library/Frameworks/R.framework/Versions/4.2-arm64/Resources/library/rticles/rmarkdown/templates/joss/resources/JOSS-logo.png}}
\fancyhead[C]{}
\fancyhead[R]{}
\renewcommand{\footrulewidth}{0.25pt}

\fancyfoot[L]{\footnotesize{\sffamily , (). phinterval: An R package for
representing and manipulating timespans with
gaps. \textit{Journal of Open Source Software}, (), . \href{https://doi.org/}{https://doi.org/}}}


\fancyfoot[R]{\sffamily \thepage}
\makeatletter
\let\ps@plain\ps@fancy
\fancyheadoffset[L]{4.5cm}
\fancyfootoffset[L]{4.5cm}

% --- Macros ---------

\definecolor{linky}{rgb}{0.0, 0.5, 1.0}

\newtcolorbox{repobox}
   {colback=red, colframe=red!75!black,
     boxrule=0.5pt, arc=2pt, left=6pt, right=6pt, top=3pt, bottom=3pt}

\newcommand{\ExternalLink}{%
   \tikz[x=1.2ex, y=1.2ex, baseline=-0.05ex]{%
       \begin{scope}[x=1ex, y=1ex]
           \clip (-0.1,-0.1)
               --++ (-0, 1.2)
               --++ (0.6, 0)
               --++ (0, -0.6)
               --++ (0.6, 0)
               --++ (0, -1);
           \path[draw,
               line width = 0.5,
               rounded corners=0.5]
               (0,0) rectangle (1,1);
       \end{scope}
       \path[draw, line width = 0.5] (0.5, 0.5)
           -- (1, 1);
       \path[draw, line width = 0.5] (0.6, 1)
           -- (1, 1) -- (1, 0.6);
       }
   }

% --- Title / Authors ---------------------------------------------------------
% patch \maketitle so that it doesn't center
\patchcmd{\@maketitle}{center}{flushleft}{}{}
\patchcmd{\@maketitle}{center}{flushleft}{}{}
% patch \maketitle so that the font size for the title is normal
\patchcmd{\@maketitle}{\LARGE}{\LARGE\sffamily}{}{}
% patch the patch by authblk so that the author block is flush left
\def\maketitle{{%
  \renewenvironment{tabular}[2][]
    {\begin{flushleft}}
    {\end{flushleft}}
  \AB@maketitle}}
\makeatletter
\renewcommand\AB@affilsepx{ \protect\Affilfont}
%\renewcommand\AB@affilnote[1]{{\bfseries #1}\hspace{2pt}}
\renewcommand\AB@affilnote[1]{{\bfseries #1}\hspace{3pt}}
\makeatother
\renewcommand\Authfont{\sffamily\bfseries}
\renewcommand\Affilfont{\sffamily\small\mdseries}
\setlength{\affilsep}{1em}


\ifnum 0\ifxetex 1\fi\ifluatex 1\fi=0 % if pdftex
  \usepackage[T1]{fontenc}
  \usepackage[utf8]{inputenc}

\else % if luatex or xelatex
  \ifxetex
    \usepackage{mathspec}
  \else
    \usepackage{fontspec}
  \fi
  \defaultfontfeatures{Ligatures=TeX,Scale=MatchLowercase}

\fi
% use upquote if available, for straight quotes in verbatim environments
\IfFileExists{upquote.sty}{\usepackage{upquote}}{}
% use microtype if available
\IfFileExists{microtype.sty}{%
\usepackage{microtype}
\UseMicrotypeSet[protrusion]{basicmath} % disable protrusion for tt fonts
}{}

\usepackage{hyperref}
\hypersetup{unicode=true,
            pdftitle={phinterval: An R package for representing and manipulating timespans with gaps},
            pdfborder={0 0 0},
            breaklinks=true}
\urlstyle{same}  % don't use monospace font for urls
\usepackage{graphicx,grffile}
\makeatletter
\def\maxwidth{\ifdim\Gin@nat@width>\linewidth\linewidth\else\Gin@nat@width\fi}
\def\maxheight{\ifdim\Gin@nat@height>\textheight\textheight\else\Gin@nat@height\fi}
\makeatother
% Scale images if necessary, so that they will not overflow the page
% margins by default, and it is still possible to overwrite the defaults
% using explicit options in \includegraphics[width, height, ...]{}
\setkeys{Gin}{width=\maxwidth,height=\maxheight,keepaspectratio}
\IfFileExists{parskip.sty}{%
\usepackage{parskip}
}{% else
\setlength{\parindent}{0pt}
\setlength{\parskip}{6pt plus 2pt minus 1pt}
}
\setlength{\emergencystretch}{3em}  % prevent overfull lines
\setcounter{secnumdepth}{0}
% Redefines (sub)paragraphs to behave more like sections
\ifx\paragraph\undefined\else
\let\oldparagraph\paragraph
\renewcommand{\paragraph}[1]{\oldparagraph{#1}\mbox{}}
\fi
\ifx\subparagraph\undefined\else
\let\oldsubparagraph\subparagraph
\renewcommand{\subparagraph}[1]{\oldsubparagraph{#1}\mbox{}}
\fi

% Pandoc syntax highlighting
\usepackage{color}
\usepackage{fancyvrb}
\newcommand{\VerbBar}{|}
\newcommand{\VERB}{\Verb[commandchars=\\\{\}]}
\DefineVerbatimEnvironment{Highlighting}{Verbatim}{commandchars=\\\{\}}
% Add ',fontsize=\small' for more characters per line
\usepackage{framed}
\definecolor{shadecolor}{RGB}{248,248,248}
\newenvironment{Shaded}{\begin{snugshade}}{\end{snugshade}}
\newcommand{\AlertTok}[1]{\textcolor[rgb]{0.94,0.16,0.16}{#1}}
\newcommand{\AnnotationTok}[1]{\textcolor[rgb]{0.56,0.35,0.01}{\textbf{\textit{#1}}}}
\newcommand{\AttributeTok}[1]{\textcolor[rgb]{0.13,0.29,0.53}{#1}}
\newcommand{\BaseNTok}[1]{\textcolor[rgb]{0.00,0.00,0.81}{#1}}
\newcommand{\BuiltInTok}[1]{#1}
\newcommand{\CharTok}[1]{\textcolor[rgb]{0.31,0.60,0.02}{#1}}
\newcommand{\CommentTok}[1]{\textcolor[rgb]{0.56,0.35,0.01}{\textit{#1}}}
\newcommand{\CommentVarTok}[1]{\textcolor[rgb]{0.56,0.35,0.01}{\textbf{\textit{#1}}}}
\newcommand{\ConstantTok}[1]{\textcolor[rgb]{0.56,0.35,0.01}{#1}}
\newcommand{\ControlFlowTok}[1]{\textcolor[rgb]{0.13,0.29,0.53}{\textbf{#1}}}
\newcommand{\DataTypeTok}[1]{\textcolor[rgb]{0.13,0.29,0.53}{#1}}
\newcommand{\DecValTok}[1]{\textcolor[rgb]{0.00,0.00,0.81}{#1}}
\newcommand{\DocumentationTok}[1]{\textcolor[rgb]{0.56,0.35,0.01}{\textbf{\textit{#1}}}}
\newcommand{\ErrorTok}[1]{\textcolor[rgb]{0.64,0.00,0.00}{\textbf{#1}}}
\newcommand{\ExtensionTok}[1]{#1}
\newcommand{\FloatTok}[1]{\textcolor[rgb]{0.00,0.00,0.81}{#1}}
\newcommand{\FunctionTok}[1]{\textcolor[rgb]{0.13,0.29,0.53}{\textbf{#1}}}
\newcommand{\ImportTok}[1]{#1}
\newcommand{\InformationTok}[1]{\textcolor[rgb]{0.56,0.35,0.01}{\textbf{\textit{#1}}}}
\newcommand{\KeywordTok}[1]{\textcolor[rgb]{0.13,0.29,0.53}{\textbf{#1}}}
\newcommand{\NormalTok}[1]{#1}
\newcommand{\OperatorTok}[1]{\textcolor[rgb]{0.81,0.36,0.00}{\textbf{#1}}}
\newcommand{\OtherTok}[1]{\textcolor[rgb]{0.56,0.35,0.01}{#1}}
\newcommand{\PreprocessorTok}[1]{\textcolor[rgb]{0.56,0.35,0.01}{\textit{#1}}}
\newcommand{\RegionMarkerTok}[1]{#1}
\newcommand{\SpecialCharTok}[1]{\textcolor[rgb]{0.81,0.36,0.00}{\textbf{#1}}}
\newcommand{\SpecialStringTok}[1]{\textcolor[rgb]{0.31,0.60,0.02}{#1}}
\newcommand{\StringTok}[1]{\textcolor[rgb]{0.31,0.60,0.02}{#1}}
\newcommand{\VariableTok}[1]{\textcolor[rgb]{0.00,0.00,0.00}{#1}}
\newcommand{\VerbatimStringTok}[1]{\textcolor[rgb]{0.31,0.60,0.02}{#1}}
\newcommand{\WarningTok}[1]{\textcolor[rgb]{0.56,0.35,0.01}{\textbf{\textit{#1}}}}

% tightlist command for lists without linebreak
\providecommand{\tightlist}{%
  \setlength{\itemsep}{0pt}\setlength{\parskip}{0pt}}


% Pandoc citation processing
%From Pandoc 3.1.8
% definitions for citeproc citations
\NewDocumentCommand\citeproctext{}{}
\NewDocumentCommand\citeproc{mm}{%
  \begingroup\def\citeproctext{#2}\cite{#1}\endgroup}
\makeatletter
 % allow citations to break across lines
 \let\@cite@ofmt\@firstofone
 % avoid brackets around text for \cite:
 \def\@biblabel#1{}
 \def\@cite#1#2{{#1\if@tempswa , #2\fi}}
\makeatother
\newlength{\cslhangindent}
\setlength{\cslhangindent}{1.5em}
\newlength{\csllabelwidth}
\setlength{\csllabelwidth}{3em}
\newenvironment{CSLReferences}[2] % #1 hanging-indent, #2 entry-spacing
 {\begin{list}{}{%
  \setlength{\itemindent}{0pt}
  \setlength{\leftmargin}{0pt}
  \setlength{\parsep}{0pt}
  % turn on hanging indent if param 1 is 1
  \ifodd #1
   \setlength{\leftmargin}{\cslhangindent}
   \setlength{\itemindent}{-1\cslhangindent}
  \fi
  % set entry spacing
  \setlength{\itemsep}{#2\baselineskip}}}
 {\end{list}}
\usepackage{calc}
\newcommand{\CSLBlock}[1]{#1\hfill\break}
\newcommand{\CSLLeftMargin}[1]{\parbox[t]{\csllabelwidth}{#1}}
\newcommand{\CSLRightInline}[1]{\parbox[t]{\linewidth - \csllabelwidth}{#1}\break}
\newcommand{\CSLIndent}[1]{\hspace{\cslhangindent}#1}



\title{phinterval: An R package for representing and manipulating
timespans with gaps}

        \author[1]{Ethan Sansom}
    
      \affil[1]{Department of Statistical Sciences, University of
Toronto}
  \date{\vspace{-5ex}}

\begin{document}
\maketitle

\marginpar{
  %\hrule
  \sffamily\small

  {\bfseries DOI:} \href{https://doi.org/}{\color{linky}{}}

  \vspace{2mm}

  {\bfseries Software}
  \begin{itemize}
    \setlength\itemsep{0em}
    \item \href{}{\color{linky}{Review}} \ExternalLink
    \item \href{}{\color{linky}{Repository}} \ExternalLink
    \item \href{}{\color{linky}{Archive}} \ExternalLink
  \end{itemize}

  \vspace{2mm}

  {\bfseries Submitted:} \\
  {\bfseries Published:} 

  \vspace{2mm}
  {\bfseries License}\\
  Authors of papers retain copyright and release the work under a Creative Commons Attribution 4.0 International License (\href{http://creativecommons.org/licenses/by/4.0/}{\color{linky}{CC-BY}}).
}

\section{Summary}\label{summary}

\texttt{phinterval} is an R (R Core Team 2022) package for representing
and manipulating time spans that may contain gaps. It implements the
\emph{phinterval} vector class, designed as an extension of the
\texttt{lubridate} (Grolemund and Wickham 2011) package's
\emph{Interval} class, to represent continuous, disjoint, empty, and
unknown spans of time.

Functionality for manipulating these spans includes:

\begin{itemize}
\tightlist
\item
  Performing set operations: union, intersection, difference, and
  complement.
\item
  Merging overlapping or adjacent intervals into non-overlapping sets of
  time spans.
\item
  Testing whether time spans, dates, or times fall within one another or
  overlap.
\end{itemize}

\section{Statement of Need}\label{statement-of-need}

Because of the complexities of accurately representing dates and times,
including adjustments for time zones, daylight saving transitions, and
leap years or seconds, manipulating time spans is a common source of
frustration and error-prone code for analysts Tiwari et al. (2025).
Several R packages, notably \texttt{lubridate} and \texttt{ivs} (Vaughan
2023), provide intuitive interfaces for representing and manipulating
time spans that handle these complexities internally, reducing the
cognitive load required of users and the likelihood of mistakes.

To the author's knowledge, however, no existing package supports empty
or discontinuous time spans. Users encountering these spans - for
example, the intersection of two non-overlapping intervals - receive an
error (as in \texttt{ivs}) or unintuitive results (as in
\texttt{lubridate}), forcing workarounds or potentially leading to
uncaught mistakes. The \texttt{phinterval} package addresses this gap by
providing explicit representations of disjoint and empty time spans,
enabling operations such as union, intersection, and set difference to
be performed accurately on arbitrary intervals. Its interface closely
mirrors that of \texttt{lubridate}, and all \texttt{phinterval}
functions accept \texttt{lubridate} \emph{Interval} vectors as inputs,
allowing analysts to safely integrate \texttt{phinterval} into their
existing workflows and work with a broader range of temporal data.

\section{Examples}\label{examples}

To demonstrate the utility of the \emph{phinterval} class, consider a
modified example from the \texttt{lubridate} package vignettes. Suppose
two colleagues are taking 5-day vacations: one to Greece in early
January and the other to Brazil in mid-February.

\begin{Shaded}
\begin{Highlighting}[]
\NormalTok{greece }\OtherTok{\textless{}{-}} \FunctionTok{interval}\NormalTok{(}\FunctionTok{ymd}\NormalTok{(}\StringTok{"2020{-}01{-}01"}\NormalTok{), }\FunctionTok{ymd}\NormalTok{(}\StringTok{"2020{-}01{-}06"}\NormalTok{))}
\NormalTok{brazil }\OtherTok{\textless{}{-}} \FunctionTok{interval}\NormalTok{(}\FunctionTok{ymd}\NormalTok{(}\StringTok{"2020{-}02{-}11"}\NormalTok{), }\FunctionTok{ymd}\NormalTok{(}\StringTok{"2020{-}02{-}16"}\NormalTok{))}
\end{Highlighting}
\end{Shaded}

If we compute the union of these intervals using the \texttt{union()}
method from \texttt{lubridate}:

\begin{Shaded}
\begin{Highlighting}[]
\FunctionTok{union}\NormalTok{(greece, brazil)}
\end{Highlighting}
\end{Shaded}

\begin{verbatim}
## [1] 2020-01-01 UTC--2020-02-16 UTC
\end{verbatim}

The resulting interval includes the intervening time between the
disjoint vacations. The \texttt{phinterval} package provides a drop-in
replacement, \texttt{phint\_union()}, which accepts the same arguments
but returns a \emph{phinterval} vector.

\begin{Shaded}
\begin{Highlighting}[]
\FunctionTok{phint\_union}\NormalTok{(greece, brazil)}
\end{Highlighting}
\end{Shaded}

\begin{verbatim}
## <phinterval<UTC>[1]>
## [1] {2020-01-01--2020-01-06, 2020-02-11--2020-02-16}
\end{verbatim}

The result is a disjoint time span, preserving the gap between the two
vacations.

In simple calculations, this distinction can easily lead to unexpected
results. For example, to calculate the number of days that either
employee is out of the office, one might take the union of their
vacation spans and then calculate the duration in days:

\begin{Shaded}
\begin{Highlighting}[]
\FunctionTok{as\_duration}\NormalTok{(}\FunctionTok{union}\NormalTok{(greece, brazil)) }\SpecialCharTok{/} \FunctionTok{ddays}\NormalTok{()}
\end{Highlighting}
\end{Shaded}

\begin{verbatim}
## [1] 46
\end{verbatim}

\begin{Shaded}
\begin{Highlighting}[]
\FunctionTok{as\_duration}\NormalTok{(}\FunctionTok{phint\_union}\NormalTok{(greece, brazil)) }\SpecialCharTok{/} \FunctionTok{ddays}\NormalTok{()}
\end{Highlighting}
\end{Shaded}

\begin{verbatim}
## [1] 10
\end{verbatim}

While experienced \texttt{lubridate} users can anticipate and work
around these cases, \texttt{phinterval} provides an intuitive
alternative which can easily be substituted into existing analyses.

\section*{References}\label{references}
\addcontentsline{toc}{section}{References}

\phantomsection\label{refs}
\begin{CSLReferences}{1}{0}
\bibitem[\citeproctext]{ref-lubridate}
Grolemund, Garrett, and Hadley Wickham. 2011. {``Dates and Times Made
Easy with {lubridate}.''} \emph{Journal of Statistical Software} 40 (3):
1--25. \url{https://doi.org/10.18637/jss.v040.i03}.

\bibitem[\citeproctext]{ref-R}
R Core Team. 2022. \emph{R: A Language and Environment for Statistical
Computing}. Vienna, Austria: R Foundation for Statistical Computing.
\url{https://doi.org/10.32614/r.manuals}.

\bibitem[\citeproctext]{ref-itsabouttime}
Tiwari, Shrey, Serena Chen, Alexander Joukov, Peter Vandervelde, Ao Li,
and Rohan Padhye. 2025. {``It's about Time: An Empirical Study of Date
and Time Bugs in Open-Source Python Software.''} In \emph{2025 IEEE/ACM
22nd International Conference on Mining Software Repositories (MSR)},
39--51. \url{https://doi.org/10.1109/MSR66628.2025.00020}.

\bibitem[\citeproctext]{ref-ivs}
Vaughan, Davis. 2023. \emph{Ivs: Interval Vectors}.
\url{https://doi.org/10.32614/cran.package.ivs}.

\end{CSLReferences}

\end{document}
